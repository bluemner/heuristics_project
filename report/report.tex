\documentclass[12pt]{article}
\usepackage[]{algorithm2e}
\usepackage{amssymb}
\usepackage{amsmath}
\usepackage[hyphens]{url}
\usepackage{listings}
\usepackage{xcolor}

\definecolor{listinggray}{gray}{0.9}
\definecolor{lbcolor}{rgb}{0.9,0.9,0.9}
\lstset{
	backgroundcolor=\color{lbcolor},
	tabsize=4,
	language=C++,
	captionpos=b,
	tabsize=3,
	frame=lines,
	numbers=left,
	numberstyle=\tiny,
	numbersep=5pt,
	breaklines=true,
	showstringspaces=false,
	basicstyle=\footnotesize,
	%  identifierstyle=\color{magenta},
	keywordstyle=\color[rgb]{0,0,1},
	commentstyle=\color{Darkgreen},
	stringstyle=\color{red}
}

\begin{document}

\title{Project 03 CS-790}
\author{Brandon Bluemner}
\date{2017}
\maketitle
% ================================================
% Abstract
% ================================================
\begin{abstract}
Algorithmic approach to the Matrix Shifting Puzzle
\end{abstract}

% =================================================
% Overview
% =================================================
\section{Overview}
\subsection{Matrix shifting Puzzle}
Let $M$ be an $n \times n$ array of numbers 1 through $n^2$,
each move $rotate(row,p)$ or $rotate(col,p)$ cost p s.t. $p \in [0...n-1]$
rotate would yield two options, rotate left or rotate right, which will 
have an impact on the out come of the algorithm. The objective is to find
the minimum cost from start state $\Rightarrow$ goal state.
\section{Implementation}
% =================================================
% Heuristic function
% =================================================
\subsection{Heuristic function}
The code implements a triangle distance Heuristic.
Trying to find the approximate distance of the diagonal.
This is done by finding the distance from the two points of the grid
start at $(current_x,current_y) \rightarrow (goal_x,goal_y)$. 
The sum absolute value of the difference with respect to the partials is
the used to get the current value. 
\begin{lstlisting}
	int cost = 0;
	int temp[2]={0,0};
	for(int i=0; i < current.size(); i++ ){
		for(int j=0; j< current.at(i).size(); j++ ){
			get_goal_position(current[i][j], goal, temp);
			int dx = temp[0]-i;
			int dy = temp[1]-j;
			cost += std::abs(dx) +  std::abs(dy);
		}
	}
	return cost;
\end{lstlisting}
% =================================================
% Hashing function
% =================================================
\subsection{Hashing}
A problem that arises with any Implementation of a board state
game is how to represent a state with the minimum amount of data.
One solution is Hashing or using a number or string to represent a state.
This Implementation uses a custom hashing method based off of Zobrist 
Hashing \cite{Zobrist}used in chess like board games. This hash code is
used to prevent recalculation of the Heuristic function an getting into
a cycle in the algorithm. 
\begin{lstlisting}
long long int get_hash_value(std::vector<std::vector<int>> &matrix){
	long long int result =0;	
	for(int i=0; i< matrix.size();i++){
		for(int j=0; j<matrix[i].size(); j++){
			auto row = (long long int)  ( (i+1) * std::pow(10, (matrix.size()-1 ) *2));
			auto col = (long long int)  ( (j+1) * std::pow(10,matrix.size()-1));
			long long int temp  = row +  col+ matrix[i][j];
			std::hash<long long int> hasher;
			auto _hash  = hasher(temp) ;
			result ^=  _hash +  0x9e3779b9   + (result<<6) + (result>>2); 
		}
	}
\end{lstlisting}
% =================================================
% Search Algorithm function
% =================================================
\subsection{Algorithm}
The algorithm used in this project was a modified version of $A*$.
Some major changes are the Implementation of hashing to improve the 
running time.This is done when the algorithm checks which nodes are in
the frontier or explored section the hashing function gets cached on
the  cpu thus causing a moderate speed gain.
% =================================================
% Sources of error
% =================================================
\section{Sources of error}
\subsection{Architecture}
Introducing hashing caused some ``error'' int the application.
This is due to how the compare method works for the minimum priority queue, 
in the event of a tie the smaller hash is used this has an impact because
each compiler will generate the $std::hash<>()$ function differently thus
causing discrepancies between each system.
\bibliographystyle{unsrt}
\bibliography{bib}

\end{document}