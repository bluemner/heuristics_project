\documentclass[12pt]{article}
\usepackage[]{algorithm2e}
\usepackage{amssymb}
\usepackage{amsmath}
\usepackage[hyphens]{url}
\usepackage{listings}
\usepackage{xcolor}

\definecolor{listinggray}{gray}{0.9}
\definecolor{lbcolor}{rgb}{0.9,0.9,0.9}
\lstset{
	backgroundcolor=\color{lbcolor},
	tabsize=4,
	language=C++,
	captionpos=b,
	tabsize=3,
	frame=lines,
	numbers=left,
	numberstyle=\tiny,
	numbersep=5pt,
	breaklines=true,
	showstringspaces=false,
	basicstyle=\footnotesize,
	%  identifierstyle=\color{magenta},
	keywordstyle=\color[rgb]{0,0,1},
	commentstyle=\color{Darkgreen},
	stringstyle=\color{red}
}

\begin{document}

\title{Project 03 CS-790}
\author{Brandon Bluemner}
\date{2017}
\maketitle
% ================================================
% Abstract
% ================================================
\begin{abstract}
Algorithmic approach to the Matrix Shifting Puzzle
\end{abstract}

% =================================================
% Overview
% =================================================
\section{Overview}
\subsection{Matrix shifting Puzzle}
Let $M$ be an $n \times n$ array of numbers 1 through $n^2$,
each move $rotate(row,p)$ or $rotate(col,p)$ cost p s.t. $p \in [0...n-1]$
rotate would yield two options, rotate left or rotate right, which will 
have an impact on the out come of the algorithm. The objective is to find
the minimum cost from start state $\Rightarrow$ goal state.
\section{Implementation}
% =================================================
% Heuristic function
% =================================================
\subsection{Heuristic function}
The code implements a relative distance.
Trying to find the approximate distance to gaol.
This is done by finding the distance from the two points of the grid
start at $(current_x,current_y) \rightarrow (goal_x,goal_y)$. 
The sum absolute value of the difference with respect to the partials is
the used to get the current value. (see Figure \ref{fig:h_func})
\begin{figure}
\begin{lstlisting}
	int cost = 0;
	int temp[2]={0,0};
	for(int i=0; i < current.size(); i++ ){
		for(int j=0; j< current.at(i).size(); j++ ){
			get_goal_position(current[i][j], goal, temp);
			int dx = temp[0]-i;
			int dy = temp[1]-j;
			cost += std::abs(dx) +  std::abs(dy);
		}
	}
	return cost;
\end{lstlisting}
\caption{Heuristic function}
\label{fig:h_func}
\end{figure}

Anther attempt was made on the same code base with the following 
change to line 8 of Figure \ref{fig:h_func} as shown in Figure \ref{fig:h_func_1}
\begin{figure}
\begin{lstlisting}
			cost +=  ((int) (std::sqrt( dx*dx + dy*dy )))
	
\end{lstlisting}
\caption{Heuristic function Change}
\label{fig:h_func_1}
\end{figure}
This yield a worst result and significate performance impact, which if you consider
the math abs can be simplified by using a bit ``hack'' (see Figure \ref{fig:bit_twiddle}) which
can eliminate computational steps compared to takeing the square root. 
\begin{figure}
	\begin{lstlisting}
	int my_abs(int x)
	{
		int s = x >> 31;
		return (x ^ s) - s;
	}
	\end{lstlisting}
	\caption{Example abs implementation}
	\label{fig:bit_twiddle}
\end{figure}
% =================================================
% Hashing function
% =================================================
\subsection{Hashing}
A problem that arises with any Implementation of a board state
game is how to represent a state with the minimum amount of data.
One solution is Hashing or using a number or string to represent a state.
This Implementation uses a custom hashing method based off of Zobrist 
Hashing \cite{Zobrist}used in chess like board games. This hash code is
used to prevent recalculation of the Heuristic function an getting into
a cycle in the algorithm. 
\begin{figure}
\begin{lstlisting}
long long int get_hash_value(std::vector<std::vector<int>> &matrix){
	long long int result =0;	
	for(int i=0; i< matrix.size();i++){
		for(int j=0; j<matrix[i].size(); j++){
			auto row = (long long int)  ( (i+1) * std::pow(10, (matrix.size()-1 ) *2));
			auto col = (long long int)  ( (j+1) * std::pow(10,matrix.size()-1));
			long long int temp  = row +  col+ matrix[i][j];
			std::hash<long long int> hasher;
			auto _hash  = hasher(temp) ;
			result ^=  _hash +  0x9e3779b9   + (result<<6) + (result>>2); 
		}
	}
\end{lstlisting}
	\caption{Hashing Code segment}
	\label{fig:Hashing}
\end{figure}
% =================================================
% Search Algorithm function
% =================================================
\subsection{Algorithm}
The algorithm used in this project was a modified version of $A*$.
Some major changes are the Implementation of hashing to improve the 
running time by eliminating checks when nodes are in
the frontier or explored section the hashing function gets cached on
the cpu and ram thus causing a moderate speed gain.
% =================================================
% Result
% =================================================
\section{Results}
The ``Path size'' will represent the transition between each state
the ``Cost'' is the cost of the transition state not including the Heuristic function weight.
\\
\subsection{2x2 matrix}
Starting position:
\begin{tabular}{ l }
	04 03 \\
	02 01
\end{tabular}
$\Rightarrow$
Goal:
\begin{tabular}{ l }
	01 02 \\
	03 04
\end{tabular}
\\ 
The results below are from a $\times5$ run on the PC(see \ref{PC}) with a 1000000 $CLOCKS PER SEC$
\\
\begin{tabular}{ | l | l | l | l | l | l | l | }
\hline
Run  & Path Size & Cost & Running Time Alg & Cpu Start & Cpu End & Total cpu time\\\hline
1 & 4 & 4 & 0.00037s & 15625 & 15625 & 0s\\\hline
2 & 4 & 4 & 0.00037s & 15625 & 15625 & 0s\\\hline
3 & 4 & 4 & 0.0003685s & 15625 & 15625 & 0s\\\hline
4 & 4 & 4 & 0.0002143s & 15625 & 15625 & 0s\\\hline
5 & 4 & 4 & 0.0002142s & 15625 & 15625 & 0s\\\hline
\end{tabular}
\\
The results below are from a $\times5$ run on Server(see \ref{Server}) with a 1000000 $CLOCKS PER SEC$
\\
\begin{tabular}{ | l | l | l | l | l | l | l | }
\hline
Run  & Path Size & Cost & Running Time Alg & Cpu Start & Cpu End & Total cpu time\\\hline
1 & 4 & 4 & 0.000589516s & 12088 & 12672 & 0.000584s\\\hline
2 & 4 & 4 & 0.000521357s & 10037 & 10553 & 0.000516s\\\hline
3 & 4 & 4 & 0.000623622s & 10185 & 10807 & 0.000622s\\\hline
4 & 4 & 4 & 0.000544579s & 9955 & 10497 & 0.000542s\\\hline
5 & 4 & 4 & 0.000579305s & 9709 & 10286 & 0.000577s\\\hline
\end{tabular}
\subsection{3x3 matrix}
Starting position:
\begin{tabular}{ l }
09 08 07 \\
06 05 04 \\
03 02 01
\end{tabular}
$\Rightarrow$
Goal:
\begin{tabular}{ l }
	01 02 03 \\
 	04 05 06 \\
 	07 08 09
\end{tabular}
\\
The results below are from a $\times5$ run on the PC(see \ref{PC}) with a 1000000 $CLOCKS PER SEC$
\\
\begin{tabular}{ | l | l | l | l | l | l | l | }
\hline
Run  & Path Size & Cost & Running Time Alg & Cpu Start & Cpu End & Total cpu time\\\hline
1 & 34 & 34 & 0.0771355s & 31250 & 109375 & 0.078125s\\\hline
2 & 34 & 34 & 0.0402095s & 15625 & 46875 & 0.03125s\\\hline
3 & 34 & 34 & 0.0326797s & 0 & 31250 & 0.03125s\\\hline
4 & 34 & 34 & 0.0325114s & 0 & 31250 & 0.03125s\\\hline
5 & 34 & 34 & 0.0324343s & 0 & 31250 & 0.03125s\\\hline
\end{tabular}
\\
The results below are from a $\times5$ run on Server(see \ref{Server}) with a 1000000 $CLOCKS PER SEC$
\\
\begin{tabular}{ | l | l | l | l | l | l | l | }
\hline
Run  & Path Size & Cost & Running Time Alg & Cpu Start & Cpu End & Total cpu time\\\hline
1 & 34 & 34 & 0.08514s & 12444 & 97607 & 0.085163s\\\hline
2 & 34 & 34 & 0.0843959s & 11669 & 96036 & 0.084367s\\\hline
3 & 34 & 34 & 0.0852655s & 11209 & 96469 & 0.08526s\\\hline
4 & 34 & 34 & 0.0837059s & 11251 & 94951 & 0.0837s\\\hline
5 & 34 & 34 & 0.0835822s & 10498 & 94075 & 0.083577s\\\hline
\end{tabular}

\subsection{4x4 matrix}


Starting position:
\begin{tabular}{ l }
16 15 15 13\\
12 11 10 09\\
08 07 06 05\\
04 03 02 01\end{tabular}
$\Rightarrow$
Goal:
\begin{tabular}{ l }
01 02 03 04\\
05 06 07 08\\
09 10 11 12\\
13 14 15 16\end{tabular}
\\ 
The results below are from a $\times5$ run on the PC(see \ref{PC}) with a 1000000 $CLOCKS PER SEC$
\\
\begin{tabular}{ | l | l | l | l | l | l | l | }
\hline
Run  & Path Size & Cost & Running Time Alg & Cpu Start & Cpu End & Total cpu time\\\hline
1 & 153 & 178 & 19.3723s & 15625 & 19375000 & 19.3594s\\\hline
2 & 153 & 178 & 19.2784s & 0 & 19281250 & 19.2812s\\\hline
3 & 153 & 178 &  19.3407s & 15625 & 19281250 & 19.2656s\\\hline
4 & 153 & 178 & 19.2833s & 0 & 19281250 & 19.2812s\\\hline
5 & 153 & 178 & 19.2716s & 0 & 19281250 & 19.2812s\\\hline
\end{tabular}
\\
The results below are from a $\times5$ run on Server(see \ref{Server}) with a 1000000 $CLOCKS PER SEC$
\\
\begin{tabular}{ | l | l | l | l | l | l | l | }
\hline
Run  & Path Size & Cost & Running Time Alg & Cpu Start & Cpu End & Total cpu time\\\hline
1 & 153 & 178 & 50.1818s & 12055 & 50126065 & 50.114s\\\hline
2 & 153 & 178 & 50.2406s & 12573 & 50170881 & 50.1583s\\\hline
3 & 153 & 178 & 50.1994s & 11521 & 50125078 & 50.1136s\\\hline
4 & 153 & 178 & 50.3866s & 12512 & 50319531 & 50.307s\\\hline
5 & 153 & 178 & 50.0151s & 11048 & 49937159 & 49.9261s\\\hline
\end{tabular}
\\
\\
The results has the same path size and cost.
From this data the algorithm is consistent with its result with path size.
There for from the data a tentative solution is provided, how ever for larger 
numbers it becomes harder to prove minimum cost. 
\subsection{other applications}
The shifting row function
was also consider in some attempts to find new algorithms for encryption \cite{rotation_enc}. 
As general Purpose graphic processing unit (GPGPU) can take advantage of basic rotation, this would allow for new forms of encryption.
\subsection{Output}
Each run generates a file, stored in the data folder, with information
about each run.

Out put file gives the hash for the start and end state \\
The Algorithm runs \\
Then it show the Visited count (explored) and the number
Iterations it went through.\\
The path is out printed next show the transitions\\
Finally the run time stats are printed.

\begin{lstlisting}
Start Hash:706246336224068
Goal Hash:706246335411732
============= Algorithm  ===========
Goal Found!!
Visited count:5
Iterations5
1	2	
3	4	
Cost:1
706246335411732->706246335412049
1	2	
4	3	
Cost:1
706246335412049->706246336219922
4	2	
1	3	
Cost:1
706246336219922->706246336224256
4	3	
1	2	
Cost:1
706246336224256->706246336224068
4	3	
2	1	
Path size: 4	Cost: 4
Running time Algorithm:	0.000589516s
cpu start: 12088	cpu end:12672	CLOCKS_PER_SEC:1000000
cpu time:	0.000584s
\end{lstlisting}
% =================================================
% Sources of error
% =================================================
\section{Sources of error}
\subsection{Compiler}
Introducing hashing caused some ``error'' into the application.
This is due to how hashing is implemented in the compiler (or windows debugger), running 
the code on windows seems to yield a less accrete result, where accuracy 
is gauge by minimum cost. 
\\
\\
Below is an out put running on PC (see \ref{PC})
\\
**Note: the running time for windows includes
the loading to the debugging library for a pdb file (as window runs the debugger)
\\
Visual C++ \textsuperscript{TM}
\begin{lstlisting}
Path size: 156	Cost: 180
Running time Algorithm:	42.9065s
cpu start: 54	cpu end:42961	CLOCKS_PER_SEC:1000
cpu time:	42.907s
\end{lstlisting}
GNU g++ 
\begin{lstlisting}
Path size: 153  Cost: 178
Running time Algorithm: 20.0533s
cpu start: 15625	cpu end:19828125	CLOCKS_PER_SEC:1000000
cpu time:	19.8125s
\end{lstlisting}
\subsection{Data structures}
This implementation only allows for n to be less then 5, if 5 is chosen
the std map library will throw a segmentation fault for exceeding size. 
If given more time this issue could have been resolved by creating custom
data structures to handle larger hash values and more combinations.
\subsection{Cpu boost}
During some of my runs the cpu on PC (See \ref{PC}) were being affected 
the the cpu boosting into higher GHz or speed. this is error is prevalent during $3\times3$
run.

% ================================
% System 
% ================================
\section{System}
\subsection{IDE}\label{IDE}
This code was programming in Visual Studio Code\cite{vscode} which an MIT ``Source Code'' editor
and debugger. which also has the 
\subsection{PC} \label{PC}
Cpu: Intel(R) Core(TM) i5-6600K CPU @3.50 GHz (stock) boost to 4.10GHz
\\
RAM: 16GB DDR4
\\
Operating System: Window 10
\subsection{Laptop} \label{Laptop}
Cpu: Intel(R) Celeron(TM)  N3150 Quad-core 1.60 GHz (stock) boost to 1.8GHz
\\
RAM: 8GB DDR3L
\\
Operating System: Window 10
\subsection{Server} \label{Server}
Cpu Intel(R) Xenon(R) CPU E5345 @ 2.33 GHZ (4 cores allocated to vm with hyper-threading)
\\
RAM: 8GB DDR2 - ECC  (Error correcting Memory)
\\
Operating System: Ubuntu Server (running on a hypervisor)
\bibliographystyle{unsrt}
\bibliography{bib}

\end{document}